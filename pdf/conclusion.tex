\section{Conclusion}

Dans le cadre de ce rapport, j'ai vu la partie théorique des CNN.\\
J'ai vu en détail comment fonctionne un réseau de neurones à convolution. En particulier, j'ai pu comprendre ce que faisaient les différents éléments de l’architecture (convolutions, pooling, relu, flattening et fully connected) et découvrir de vrais réseaux utilisés en production (LeNet, VGG et GoogLeNet).\\[1cm]
Afin de compléter la partie théorique, et sachant le travail et les recherche qu'il faut accomplir, ainsi le nombre de compétences qu'il faut acquérir j’envisage de savoir plus sur ces réseaux de neurones à convolution afin d’implémenter différentes architectures de ce dernier et de pouvoir les comparer aux autres réseaux de neurones.